\documentclass{article}

% general packages:
\usepackage[utf8]{inputenc}		% use Unicode
\usepackage{tipa} 				% for IPA
\usepackage{tikz}				% for Diagrams
	\usetikzlibrary{positioning}       	% for relative positioning 
\usepackage{graphicx}			% for inserting images
	\graphicspath{ {./} }			% the path for the images
\usepackage{hyperref}			% for links

% packages for Dynamic OT:
\usepackage{pifont}			         % for pointing hand
\usepackage{arydshln}			% for dashed lines
\usepackage{rotating}			% for angled text

% more packages:
\usepackage{multirow} 			% for tables
\usepackage{enumitem} 			% for ordered list a-b-c 
\usepackage{fancyhdr}			% for header in each page
\pagestyle{fancy}

\begin{document}	

\begin{titlepage}
    \begin{center}
	\vspace*{2cm}
        	\huge
	% use the word "evolutionary"
	% simulated annealing - "SA", genetic algorithm - "GA"
        	\textbf{Learning Head-Complement Order with Minimalist Grammars \newline \\ Using a genetic algorithm}


            
        	\vspace{2cm}
        	\LARGE
        	\textbf{Roy Mayan}\\
       	Submitted to: Prof. Roni Katzir 
            
        	\vfill
          
	Presented as a final paper of\\
	Parsing Seminar: Computation and Cognition
            
        	\vspace{0.8cm}
            
        	\Large
       	Department of Linguistics\\
       	Tel Aviv University\\
        	July 2024
    \end{center}
\end{titlepage}

%I wish to thank the people who participated in my study,  Clémence who helped my with the execution of the experiment, and my dear friends Hila and Maya for their input and proofing of the paper. 

\section*{Abstract}
{\sffamily\small

Head-complement order varies between and within languages. Several theories have been proposed to describe how this might be represented mentally. However, all current theories agree that some element of it must be learned – either the order itself or some sort of appropriate movement.
These theories can be explored and compared using a computational learning perspective, and first steps toward this comparison were made by Avraham (2017). However, current results show the learner prefers the "wrong" grammar over the expected grammar.

In this paper (which compremises the first part of my seminar paper), I implemented a genetic based algorithm for Avraham's learner, providing a different approach for finding the best grammar when learning (as opposed to Simulated Annealing).
The purpose is to improve the learning process, and have the learner arrive at the expected grammar.  Switching from Simulated Annealing to a genetic algorithm is the initial step in this direction.
% step one - genetic instead of simulated annealing
% step two - a learner that works in several steps
% step three - adding more syntactic structures in addition to coordination (topicalization & questions)
% step four - frequency analysis
This enables us to advance to the next phase, where comparisons can be made between the different theories.
%The results show.
}

\section{Introduction} % 1 - Introduction
Schwa deletion in French is a phonological pattern that has been extensively researched in phonological literature. A specific instance of such deletion was presented by Dell (1973) in his famous book "Les règles et les sons" (The rules and sounds) under "Schwas en syllabes contiguës" (Schwas in contiguous syllables) among many other cases of schwa deletion. This specific phenomenon is known as Optional Schwa Deletion (henceforth OSD), and has been examined several times later on,  challenging various phonological theories. A question that arises is whether native speakers of French approve/disprove of the different options of deleting the schwa in some cases as Dell suggested.\\

The goal of this paper is to try to answer this question by conducting an experiment, in order to determine whether Dell's findings are characteristic of  French speakers today. 
In section 2, I introduce the theoretical background for my work (Head-Complement order and previous work),  then I present my approach for improving the learner and explain how I address it. 
Then,  in section 3,  I present the implementation of the genetic algorithm. 
Finally,  in section 4,  I summarize the conclusions that I draw from my work.

\clearpage
\section{Background} % 2 - Background
In linguistics, the study of languages, phonology is the study of their pronunciation,  the sound system of a language and its properties.

\subsection{Terminology and basic definitions} % Head-Complement order, Theoretical Models for Learning, UG
\begin{enumerate}

  \item \textbf{Universal Grammar (UG)}: The premise of Generative Linguistics is the fact that native speakers have knowledge about
their language that allows them to produce and comprehend words and sentences that they have never heard before. The description of that knowledge is called a Grammar, and it is what humans are assumed to acquire when acquiring language. This led to the thought that humans are born with some characterization of the set of grammars that they are able to acquire,  which Chomsky proposed to call UG.
% is it necessary???
  \item Other definitions: Dell uses the sign \# to mark the boundaries of a word, and * for incorrect derivations.

\end{enumerate}


% following subsections
\subsection{Current implementation (SA)}
\subsection{Genetic algorithms (GA)}
% (i) Genetic algorithms (what is it? what is the general structure?)
% (ii) Why choose GA/Use of evolutionary algrithms in statistical NLP bla bla (advantages, why did I choose to go in this direction? based on various articles I have read - צריך לעשות פה סקירות ספרות קטנה ולשכנע למה אני מנסה לממש כזה דבר))

\subsection{OSD in French}
% intro
The vowel transcribed as [ə] in French is a mid front rounded vowel whose specific realization varies across contexts and dialects.  It has been the subject of a large body of research,  because (among other reasons) it is optional in a variety of positions.  In this paper I discuss a specific context that Dell suggested in 1973 in his book translated as \textit{Generative phonology and French phonology}.  A big part of that book is devoted to a detailed discussion of the behaviour of schwa. Among many cases of schwa deletion, Dell presented this case of OSD as follows, starting with the following rule:\\

% present the rule, and examples
\begin{center}
\textbf{VCE (optional): ə → ø / V\#C\_} \\
(meaning, a schwa is deleted after a vowel, word boundary and a consonant)
\end{center}
When several successive syllables contain schwas likely to be erased by VCE, a speaker speaking at a normal rate tends to drop the maximum number of schwas possible.  This is a trend rather than an absolute necessity. \\ 
Let's consider \textit{(tu as) envie de te battre} "(you) feel like fighting",  represented as [\textipa{\~avi\*\#d@\*\#t@\*\#batr}].
According to Dell, we can decide for each environment whether to delete the schwa or not, and that's how we get the following options:
\begin{enumerate}
  \item Output is [\textipa{\~avi\*\#d@\*\#t@\*\#batr}] - no deletion
  \item Output is [\textipa{\~avi\*\#d\*\#t@\*\#batr}] - the first schwa is deleted, but not the second 
  \item Output is [\textipa{\~avi\*\#d@\*\#t\*\#batr}] - the second schwa is deleted, but not the first 
  \item Output is *[\textipa{\~avi\*\#d\*\#t\*\#batr}] - deleting both schwas is not possible
\end{enumerate}
% self-bleeding, left-to-right
We will analyze the application of the rule left-to-right:
\begin{itemize}
  \item Given the basic form [\textipa{\~avi\*\#d@\*\#t@\*\#batr}],  we can first decide to delete the first schwa or not, because the environment is met.
  \item If we apply it, we get option 2 - and then the environment for the second schwa is not met and we have no more options.
  \item If not, we continue to the second schwa and we can again decide whether to delete or not.
  \item If we apply the rule, we get option 3, otherwise we stay with option 1 (the original one).
  \item We don't generate option 4 - and that is the wanted situation.
\end{itemize}

\subsection{OSD's Significance} % Iterativity, local optionality
As mentioned, OSD has several interesting attributes that gained attention in literature.  I would like to focus on two important appearences of OSD in phonological history: the first is related to Iterative rule application (as an argument for self-bleeding),  and the second is in the field of OT and how it deals with local optionality. I will give a brief explanation on each context in which OSD was mentioned, and focus on its properties that made it relevant to those contexts.\\

\begin{description}
   \item[Iterativity] This example from French was used as an argument against a theory of Chomsky \& Halle (1968) which declared that rules operate non-iteratively: that is, if a specific rule has several environments in which it can apply, it would apply to all of them together - simultaneously.  The prediction of a non-iterative application is no self-feeding/self-bleeding interactions. The reason OSD is an evidence against their theory, is due to the fact that here there seems to be an interaction of \textit{self-bleeding}: if the rule applies in one position, it ruins its potential environment to apply in a second position. If the rule were to apply simultaneously on both environments in the input, it would apply twice - and that is not what actually happens (according to Dell). It appears to be that if there are several environments, the deletion must take place one after another, so that the rule can bleed itself and generate the correct output.
\end{description}

\subsection{Research Question}
As previously stated, the data regarding OSD which was cited many times later, comes from Dell's work. As of now, we do not know exactly to what extent his generalisations are correct, and whether French speakers do prefer these patterns in different combinations of words.

My main research question is whether speakers of French are willing to accept the patterns Dell suggested in his analysis of OSD back in 1973.  Do they prefer the opaque (caused by self-counterbleeding) or transparent (caused by self-bleeding) forms?

Another question is related to the direction in which the rule applies. As mentioned,  Dell's [kstr] example does favor left-to-right, but is this what happens in most cases? Are there any counter-examples?\\

The importance of knowing whether Dell's suggestions are characteristic of speakers today is directly related to its historical significance in the discussion over different theories of phonology,  as explained above (the connection between iterativity and opacity, locality, optionality) - that is the motivation for this study. I will try to address this question using an experiment, as described in detail in the following section.


\clearpage
\section{Implementation} % 3 - Main: implementing GA
In order to test native French speakers' knowledge of their language to verify Dell's analysis, they were exposed to recordings of different pronunciation options and asked to rate them.  For online data collection,  the software PsyToolkit was used (Stoet, 2010, 2017). After constructing the basic phrases myself, I consulted a French speaker who also recorded in her voice all the different versions as described below. 


% How do I implement a GA here? How do I define the different objects needed?
% How do I incorporate it in the given code base?
% How does it affect the MDL metric? is it no longer needed?

\subsection{Design}
While designing the experiment, my goal was to make it as simple and concise as possible, to obtain more engagement from people completing it fully. Therefore, I decided to define the following conditions:
\begin{itemize}
  \item 4 main phrases.
  \item About 3-5 different pronunciations of each.
  \item The phrases contain only 2 schwas.
\end{itemize}
% Phrases (a)-(b)
First,  I constructed 2 phrases based on the [kstr] case. The first one is the original phrase presented by Dell - \textit{il veut que ce travail soit bien fait}, where \textit{que} and \textit{ce} are the source for the schwas, leading to the sequence: [\textipa{v\"o\*\#k@\*\#s@\*\#tra}].
The second phrase is based on the same principle but with a different sequence: [kspr]. Using a French corpus I found online (\cite{corpus}), I was searching for similar sequences to [kstr] (that appears in the word \textit{extraordinare}.) I came up with [dstr] as an option, but could not find any French words containing it. I found the sequence [kspr] (orthographically "xpr") such as in the verb \textit{exprimer} "to express", which meets the conditions. I constructed the phrase \textit{elle veut que ce produit soit parfait} "she wants this product to be perfect". We have:
\begin{enumerate} [label=(\alph*)]
  \item \textit{il veut que ce travail soit bien fait} with [\textipa{v\"o\*\#k@\*\#s@\*\#tra}]
  \item \textit{elle veut que ce produit soit parfait} with [\textipa{v\"o\*\#k@\*\#s@\*\#pro}]
\end{enumerate}
For both (a) and (b), I wanted to record the 3 basic options: no deletion, left-schwa deleted, right-schwa deleted. Since these phrases can be pronounced easily with the 2 schwas deleted,  I included this version to be recorded. I believed it would be interesting to compare between the fully deleted version, and a sentence containing the sequence as found in the language - were it should be perfectly grammatical and approved.  For this reason I added two extra phrases that contain \textit{extraordinare} and \textit{exprimer}:
\begin{itemize}
  \item \textit{c’est \textbf{extraordinaire} ce qu’il fait} "it's amazing what he does"
  \item \textit{je peux \textbf{m’exprimer} sans begayer} "I can express myself without stuttering"
\end{itemize}

% building 
\subsection{Experiment}
The pairs of recordings were formed as 15 questions in the online survey, where for each pair of recordings "A" and "B" there was a seperate scale: from 1 (pas d'accord) to 7 (d'accord) - that is, 1 is the lowest grade of acceptance of the pronunciation, and 7 is the highest grade. After having built the full survey,  I shared it online using the help of French speakers, who forwarded it to more people.  
Regarding the responses I received:
\begin{enumerate}
  \item A total of 22 subjects.
  \item All the subjects are native French speakers - mostly from France, but also from Belgium, Israel,  and Asia.
  \item The subjects’ ages ranged from 21 to 65 with an average of 29.22.
\end{enumerate}

\subsection{Expectations}
From conversations I had with about 3-4 French speakers regarding the OSD issue whilst preparing the experiment,  I got the impression that on the whole they agree with Dell's proposition; The default option is clearly valid. Deleting either schwa is optional and completely valid (though considered as informal speech), and the tendency to delete the left or right schwa depends on the specific French dialect, region etc. In this humble experiment I ignored these factors (which were researched for example by Bayles \& Kaplan (2016)), and focused on the general acceptance of the recordings with random French speakers. 
Regarding the fully deleted versions - most speakers agreed they sound bad, though one pointed out he thinks recording 4 is not completely wrong (he can imagine French speakers producing it, and stated he would understand the meaning with no effort). 

Therefore, I expect a general correlation with Dell's analysis: phrases 1-3,6-8,11-13,15-17 being correct. In addition, I expect some variation across phrases 4 and 9 which can be easily pronounced, and maybe not completely wrong as Dell treated them. 

\clearpage
\section{Conclusions} % 4 - Conclusions

\subsection{Results and analysis}
There are many ways to analyze the results of the experiment. Given the clear division into pairs using the previously defined categories (N, R, L , D) it will be interesting to see the results of the same categories across different phrases.  Also,  I'm interested in looking for the amount of consistency regrading the same pair/phrase.  Here are the main points that came up from the data:
\begin{enumerate}
  \item The N version (no deletion at all) given in phrases 1,5,6,10,11,15 got the average score of \textbf{6.6} as expected.
\end{enumerate}

\subsection{Possible problems and recommendations}
\begin{enumerate}
  \item I chose sentences with only 2 schwas, but it will be interesting to test more complicated cases with 3 or even 4 schwas (such examples were given by Dell himself).
\end{enumerate}

\subsection{Discussion}
The OSD pattern in French is an interesting phenomenon due to its popularity and significance in literature. The generalizations given by Dell to explain the OSD logic seem to be characteristic of French speakers today. The results of this small experiment I conducted support Dell's propositions, hence they support the analysis of the VCE rule application as left-to-right. Speakers seem to have strong intuitions about this phenomenon.
My conclusions are certainly tentative because of the previously mentioned confounds in the experiment,  that should be addressed in a more extensive follow-up experiment (given my recommendations).

\clearpage
\fancyhead{} % clear all header fields
\fancyhead[RO]{\textit{REFERENCES}}

\begin{thebibliography}{} % - References

% section 1
\bibitem{texbook} 
Avraham, T. (2017). 
Learning Head-Complement Order with Minimalist Grammars.
MA thesis, Tel Aviv University.


% section 2
\bibitem{texbook}
Stabler, E. (1996). 
Derivational minimalism. In International conference on logical aspects of computational linguistics (pp. 68-95). 
Berlin, Heidelberg: Springer Berlin Heidelberg.


\end{thebibliography}

\end{document}